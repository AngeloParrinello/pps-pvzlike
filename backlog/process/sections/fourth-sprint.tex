\section*{Fourth Sprint Planning - 23/10/2022}
Il team si è incontrato via Microsoft Teams per definire gli obiettivi del quarto sprint.

\subsection*{Obiettivi}
\begin{itemize}
    \item Polish aspetti grafici (altezza del colpo, qualità degli assets, posizione di piazzamento delle torrette, \ldots)
    \item Creare tipi diversi di truppe.
    \item Pulsanti per mettere in pausa, riprendere e modificare la velocità della partita.
    \item Realizzazione della schermata di fine partita.
    \item Realizzazione della schermata di inizio partita (Opzionale).
    \item Generazione delle wave con Prolog.
\end{itemize}

\subsection*{Planning}
Deadline sprint: 3 Novembre.

\subsubsection*{Suddivisione del lavoro}
\begin{itemize}
    \item Parrinello, Penazzi => Generazione delle wave con Prolog.
    \item Foschini, Penazzi => Creare tipi diversi di truppe.
    \item Parrinello => Realizzazione della schermata di fine partita.
    \item Foschini => Realizzazione della schermata di inizio partita (Opzionale).
    \item Alpi, Foschini => Pulsanti per mettere in pausa, riprendere e modificare la velocità della partita.
    \item Alpi => Polish aspetti grafici (altezza del colpo, qualità degli assets, posizione di piazzamento delle torrette, \ldots).
    \item Alpi, Foschini => Realizzazione delle animazioni delle entità (Opzionale).
    \item All => Realizzazione del gioco completo, compreso di menù di fine partita.
\end{itemize}

\subsection*{Sprint Review - 03/11/2022}
Risultati:
\begin{itemize}
    \item La teoria per la generazione delle wave con Prolog è stata fatta, manca però l'integrazione con Scala.
    \item Sono stati creati diversi tipi di truppe.
    \item La schermata di fine partita è stata creata, così come quella di inizio.
    \item Gli aspetti grafici sono stati migliorati
    \item Il gioco risulta completo.
\end{itemize}
Non sono state realizzate le animazione ma erano state definite opzionali.
Non sono stati realizzati i pulsanti per mettere in pausa e riprendere la partita a causa del poco tempo.
Il team si ritiene soddisfatto in quanto una versione completa del gioco è stata realizzata, nonostante sia ancora migliorabile in certi aspetti.
Rispetto agli sprint precedenti, il team è riuscito a fare previsioni più accurate sul carico di lavoro necessario per raggiungere gli obiettivi prefissati.
