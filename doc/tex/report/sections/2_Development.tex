\section{Processo di sviluppo}
\label{sec:development}
Il processo di sviluppo adottato dal team è ispirato a Scrum: sarà basato su \textbf{sprint} e \textbf{obiettivi}
per realizzare il progetto in maniera \textbf{agile}.
All'interno del team sono stati scelti un \textit{committente} e un \textit{product owner}.
Il team effettua sprint della durata di circa due settimane, durante i quali si definiscono gli obiettivi e si suddividono i compiti.
Di seguito, si discutono gli elementi fondamentali del processo di sviluppo adottato.

\subsection{Meeting}
I meeting sono un fattore fondamentale per il processo di sviluppo e avvengono con cadenza quasi giornaliera
e con durate differenti in base all'importanza dello stesso. Vista la distanza tra i membri del
team (due in Svezia e due in Italia) la maggior parte degli incontri è avvenuto a distanza,
utilizzando *Microsoft Teams* come piattaforma principale.

\subsection{Modalità di divisione in itinere dei task}

\subsubsection{Definition of Done}
Una funzionalità di gioco viene definita completata quando, a seguito di una revisione da parte
di un altro componente del team, viene pubblicata sul \textit{branch} principale. Questa revisione può essere avvenuta
tramite \textit{pair-programming} oppure il meccanismo di \textit{pull-request}.

\subsubsection{Coordinazione}
La comunicazione è fondamentale per un processo di sviluppo agile, anche se i membri del team si conoscono a fondo.
Per coordinarsi al meglio, il team ha deciso di utilizzare \textbf{Trello},
con il quale vengono tracciati i task dei singoli membri con il rispettivo andamento,
individuando un flusso di lavoro all'interno di ogni sprint organizzativo.
Inoltre, il \textit{product owner} del team ha redatto un \textit{product backlog} nel
quale si è tenuto traccia dei task portati a termine da ciascun membro del gruppo, indicando
per ciascuno il costo in termini di tempo e difficoltà di progettazione e/o implementazione.

\subsubsection{Meeting iniziale}
Prima della proposta del progetto, è avvenuto un incontro all'interno del quale sono stati decisi i seguenti fattori essenziali:
\begin{itemize}
    \item \textbf{Ruoli}: colui che ha proposto il progetto è stato eletto committente, mentre è stato scelto product owner il membro con più esperienza in Scrum.
    \item \textbf{Specifiche}: sono stati decisi gli obiettivi funzionali, facendo attenzione alla loro fattibilità.
    \item \textbf{Primo Sprint}: è stata decisa l'organizzazione del primo sprint, definendo l'obiettivo finale e suddividendo dei *task* tra i componenti del gruppo.
\end{itemize}
La durata del primo meeting è stata di circa 2 ore.

\subsubsection{Sprint Planning}
All'inizio di ogni sprint viene effettuato un incontro all'interno del quale si discutono i
risultati dello sprint precedente e si definiscono gli obiettivi di quello successivo.
I principali punti sui cui ci si focalizza sono i seguenti:
\begin{itemize}
    \item Definizione degli obiettivi.
    \item Definizione ed assegnazione dei task.
    \item Valutazione dell'andamento complessivo del progetto, rimarcando eventuali ritardi.
    \item Valutazione dello sprint precedente.
\end{itemize}

La durata ideale degli sprint planning è fissata a 2 ore.

\subsubsection{Divisione dei compiti}
L'effettiva divisione dei compiti, da eseguire nello sprint successivo all'interno del team, viene fatta
contestualmente alla chiusura dello sprint precedente. La suddivisione dei compiti terrà conto del carico di lavoro, degli
impegni del singolo componente e dell'eventuale lavoro incompiuto dallo sprint precedente.

\subsection{Modalità di revisione in itinere dei task}

\subsubsection{Ridistribuzione del carico lavorativo}
All'interno dello sprint è previsto la ridistribuzione del lavoro. Infatti, se ci si accorge che il carico di lavoro
di un task risulta essere diverso da quello previsto riteniamo possibile ridefinire i partecipanti al suddetto task.
Questo bilanciamento dovrà avvenire al seguito di un \textit{Stand-up Meeting} e di una valutazione da parte dell'intero
gruppo.

\subsubsection{Stand-up Meeting}
Con cadenza quasi giornaliera, il team effettua degli incontri in cui ognuno dei membri
espone il lavoro portato a termine, dichiarando eventuali difficoltà.
La durata ideale degli stand-up meeting è fissata a 10/15 minuti.

\subsection{Tool Ausiliari}
A supporto del processo agile, il team si impone di utilizzare strumenti con lo scopo di migliorare
l’efficienza e di consentire al gruppo di concentrarsi maggiormente sul processo di sviluppo.

\subsubsection{Automazione}
Sono stati adoperati i seguenti processi:
\begin{itemize}
    \item \textbf{Test Driven Development} - scrivere in anticipo dei test per il proprio codice,
    permette di intercettare eventuali errori che insorgono durante l'integrazione del lavoro reciproco.
    \item \textbf{Continuous Integration} - per verificare la compatibilità e la correttezza del software prodotto
    viene sfruttata la funzionalità \textbf{GitHub Actions}, definendo all'inizio del processo di sviluppo un
    preciso workflow che assicura la corretta manutenzione del progetto.
    \item \textbf{Continuous Delivery} - per evitare di produrre manualmente gli \textit{artifacts}
    alla fine di ogni Sprint, è stato individuato un workflow adatto che permettesse di creare delle
    release automatiche ogni qual volta il branch \textit{release} di git venisse aggiornato.
    \item \textbf{Automatic Dependecies Updates} - per mantenere aggiornate le dipendenze del codice sorgente, attraverso
    delle pull request, verrà impiegato il bot \textbf{Renovate}.
    \item \textbf{Code Quality Automation} - attraverso il tool \textbf{Sonarcloud} sarà possibile automatizzare la ricerca
    di segnali di cattiva qualità del codice quali \textit{code repetition}, \textit{bugs} e \textit{vulnerabilità}.
\end{itemize}

\subsubsection{Versioning}
Per gestire in maniera ottimale il codice prodotto, si è deciso di impiegare \textbf{Git} già sperimentato da tutti i
componenti del gruppo. Allo stato attuale, il gruppo non ha intenzione di adottare la specifica del Semantic Versioning.
Si ritiene però opportuno l'adozione dei \textit{Conventional Commits} per due motivi: i singoli commit saranno più chiari
e in futuro si potrebbe decidere di impiegare il Semantic Versioning.

