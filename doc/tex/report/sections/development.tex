\section{Processo di sviluppo}
\label{sec:development}
Il processo di sviluppo adottato dal team è ispirato a Scrum: sarà basato su \textbf{sprint} e \textbf{obiettivi} per realizzare il progetto in maniera \textbf{agile}.
All'interno del team sono stati scelti un \textit{committente} e un \textit{product owner}.
Il team effettua sprint della durata di circa due settimane, durante i quali si definiscono gli obiettivi e si suddividono i compiti.
Di seguito, si discutono gli elementi fondamentali del processo di sviluppo adottato.

\subsection{Meeting}
I meeting sono un fattore fondamentale per il processo di sviluppo e avvengono con cadenza quasi giornaliera e con durate differenti in base all'importanza dello stesso. Vista la distanza tra i membri del team (due in Svezia e due in Italia) la maggior parte degli incontri è avvenuto a distanza, utilizzando *Microsoft Teams* come piattaforma principale.

\subsubsection{Meeting iniziale}
Prima della proposta del progetto, è avvenuto un incontro all'interno del quale sono stati decisi i seguenti fattori essenziali:
\begin{itemize}
  \item \textbf{Ruoli}: colui che ha proposto il progetto è stato eletto committente, mentre è stato scelto product owner il membro con più esperienza in Scrum.
  \item \textbf{Specifiche}: sono stati decisi gli obiettivi funzionali, facendo attenzione alla loro fattibilità.
  \item \textbf{Primo Sprint}: è stata decisa l'organizzazione del primo sprint, definendo l'obiettivo finale e suddividendo dei *task* tra i componenti del gruppo.
\end{itemize}
La durata del primo meeting è stata di circa 2 ore.

\subsubsection{Sprint Planning}
All'inizio di ogni sprint viene effettuato un incontro all'interno del quale si discutono i risultati dello sprint precedente e si definiscono gli obiettivi di quello successivo. I principali punti sui cui ci si focalizza sono i seguenti:
\begin{itemize}
  \item Definizione degli obiettivi.
  \item Definizione ed assegnazione dei task.
  \item Valutazione dell'andamento complessivo del progetto, rimarcando eventuali ritardi.
  \item Valutazione dello sprint precedente.
\end{itemize}

La durata ideale degli sprint planning è fissata a 2 ore.

\subsubsection{Stand-up Meeting}
Con cadenza quasi giornaliera, il team effettua degli incontri in cui ognuno dei membri espone il lavoro portato a termine, dichiarando eventuali difficoltà.
La durata ideale degli stand-up meeting è fissata a 10/15 minuti.

\subsection{Tool Ausiliari}
A supporto del processo agile, il team si impone di utilizzare strumenti con lo scopo di migliorare l’efficienza e di consentire al gruppo di concentrarsi maggiormente sul processo di sviluppo.

\subsubsection{Automazione}
Sono stati adoperati i seguenti processi:
\begin{itemize}
    \item \textbf{Test Driven Development} - scrivere in anticipo dei test per il proprio codice, permette di intercettare eventuali errori che insorgono durante l'integrazione del lavoro reciproco.
    \item \textbf{Continuous Integration} - per verificare la compatibilità e la correttezza del software prodotto viene sfruttata la funzionalità \textbf{GitHub Actions}, definendo all'inizio del processo di sviluppo un preciso workflow che assicura la corretta manutenzione del progetto.
    \item \textbf{Continuous Delivery} - per evitare di produrre manualmente gli \textit{artifacts} alla fine di ogni Sprint, è stato individuato un workflow adatto che permettesse di creare delle Release automatiche ogni qual volta il branch principale di git (\textit{main}) venisse aggiornato.
\end{itemize}

\subsubsection{Coordinazione}
La comunicazione è fondamentale per un processo di sviluppo agile, anche se i membri del team si conoscono a fondo.
Per coordinarsi al meglio, il team ha deciso di utilizzare \textbf{Trello}, con il quale vengono tracciati i task dei singoli membri con il rispettivo andamento, individuando un flusso di lavoro all'interno di ogni sprint organizzativo. Inoltre, il \textit{product owner} del team ha redatto un \textit{product backlog} nel quale si è tenuto traccia dei task portati a termine da ciascun membro del gruppo, indicando per ciascuno il costo in termini di tempo e difficoltà di progettazione e/o implementazione.
