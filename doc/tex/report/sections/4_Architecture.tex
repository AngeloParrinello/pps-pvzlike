\section{Architettura}

\subsection{Pattern Architetturale}
Il pattern architetturale scelto per questo progetto è il celebre \textbf{MVC}.
Questa scelta ha portato ad una divisione netta dei compiti tra i vari componenti. In fase di design il pattern si
è rivelato importante perchè ci ha permesso di dividere in maniera ottimale il flusso di lavoro e i task che ognuno dei
componenti doveva portare a termine. Questa scelta, unita al \textit{Modello ad Attori}, ha aiutato
lo sviluppo permettendoci di astrarre alcuni concetti e rendere il più indipendenti possibili i vari componenti del software.

\subsection{Modello ad Attori}
La presenza di un elevato numero di entità all’interno dell’applicazione ha portato alla luce
l’esigenza di gestire, già a livello architetturale, la possibilità di sfruttare a pieno la potenza
della cpu rendendo quindi il programma concorrente. In particolare, per astrarre dalla gestione
di problematiche tipiche di sistemi concorrenti (mutua eclusione e corse critiche) si è deciso di
adottare un approccio basato su attori. Dal momento che ogni attore incapsula al suo interno
un flusso di controllo basato su \textit{event loop} e che le interazioni avvengono esclusivamente
attraverso scambio di messaggi, non è necessario gestire meccanismi di sincronizzazione.
Inoltre, abbiamo ritenuto il modello ad attori particolarmente adeguato per la realizzazione
dell’applicazione in quanto essi incapsulano un \textbf{comportamento}, il quale ben si adatta a
descrivere l’effettivo comportamento delle entità all’interno del modello. Tale scelta ci ha
quindi garantito un livello di astrazione ed un’espressività che ha positivamente influenzato
anche le fasi successive di design ed implementazione. Per tali motivi si è deciso quindi di adottare il framework
\textit{Akka} per sviluppare in maniera semplificata il sistema ad attori.

\subsubsection{Vantaggi del Modello ad Attori}
Di seguito si analizzano i vantaggi riscontrati nello sviluppo del sistema a fronte dell'utilizzo di Akka:
\begin{itemize}
    \item Chiarezza del comportamento del sistema fin da subito.
    \item Espressività del codice relativo al comportamento delle entità di gioco.
    \item Migliore incapsulamento dei task di ogni componente.
    \item Sistema maggiormente modulare.
    \item Astrazione di meccanismi di concorrenza di basso livello (semafori, monitor, ecc.).
\end{itemize}

\subsubsection{Svantaggi del Modello ad Attori}
La scelta del paradigma ad attori ha portato tanti benefici ma sono stati individuati i seguenti svantaggi:
\begin{itemize}
    \item Difficoltà nel testing degli attori.
    \item Il numero dei messagi scambiati è proporzionale alla dimensione del sistema.
    \item Eventuale \textit{bottleneck} dovuto al message passing.
\end{itemize}

\subsection{Architettura Complessiva}
L'impiego del paradigma ad attori combinato al pattern architetturale MVC ci ha portato a modellare questi tre componenti
come essi stessi attori. Ognuno di essi gestisce messaggi di tipo diverso:
\begin{itemize}
    \item
\end{itemize}