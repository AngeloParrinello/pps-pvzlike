\newpage
\section{Implementazione}


\subsection{Francesco Foschini}
Il mio compito all'interno del progetto, è stato quello di modellare la gerarchia di \texttt{Entity} del gioco.
In particolare mi sono occupato della implementazione del model degli Zombie, e i proiettili.
Ho partecipato inoltre allo sviluppo del Troop Actor del Menù Screen e del GameScreen.

\subsubsection{Bullet}
I proiettili rappresentano all'interno del gioco l'attacco delle piante e degli zombie che viene sparato dalle torri
per infliggere danni. Grazie alla buona gerarchia di caratteristiche definite in \texttt{Entity}, è stato possibile grazie  ai \textbf{mixin}
dotare il trait \texttt{Bullet} dell'abilità di movimento (\texttt{Movement Ability}).
Il trait Bullet rappresenta quindi l'entità base "sparata" da una Troop.
Tutti i bullet presentano un comportamento comune: possiedono una posizione, una velocità, una dimensione ed un danno.


Successivamente a una prima versione dei Bullet ho pensato di rifattorizzare i Bullet individuando due diversi sottotipi: \texttt{ZombieBullet} e \texttt{PlantBullet}.
Il primo è un trait che modella l'attacco base degli zombie mentre il secondo si riferisce all'attacco delle piante.
Attraverso il metodo checkCollision i PlantBullet possono collidere solamente contro le entità di tipo Zombie mentre,
al contrario, gli ZombieBullet potranno colpire solamente le entità sottotipo di Plant.
Mi sono successivamente concentrato nell'implementazione di due tipi di ZombieBullet: PawBullet (il bullet dei
BasicZombie e FastZombie) e lo SwordBullet(il bullet dei WarriorZombie).
Lo SwordBullet è un attacco molto più potente rispetto al PawBullet.

Volendo seguire un approccio puramente funzionale il metodo di aggiornamento della posizione di un bullet deve crearne uno nuovo
con i valori aggiornati. Questo comportamento è comune a tutti i bullet, ma l'aggiornamento
di ognuno di questi deve tornare il tipo di bullet corretto. A tal proposito è stato utilizzato il metodo
di libreria di scala  \texttt{copy()}.

La velocità e il danno di ogni tipo di bullet è definito nell'oggetto BulletDefaultValues.

\begin{figure}[H]
    \centering
    \includegraphics[width=1\linewidth]{img/class-bullet}
    \caption{Diagramma delle classi rappresentante i \texttt{ZombieBullet}.}
    \label{fig:class-bullet}
\end{figure}

\subsubsection{Zombie}
Gli Zombie rappresentano una delle entità fondamentali del gioco. In particolare, rappresentano i nemici che l'utente
deve cercare di eliminare attraverso il piazzamento delle piante nel campo di gioco.
Tutti gli zombie sono in grado di muoversi "orrizzontalmente" nelle lane del campo di gioco con una certa velocità: pertanto
attraverso il meccanismo dei mixin il \texttt{trait Zombie} estende il \texttt{trait Troop} a cui aggiunge poi l' abilità \texttt{MovingAbility}.

Quando uno zombie è colpito da un CherryBullet potrebbe essere rallentato, per questo motivo si è deciso di
mettere la velocità in ciascun costruttore per ciascun tipo di zombie implementato.
In questo modo, seguendo un approccio funzionale, quando il CherryBullet collide con uno specifico Zombie
istanzierà nuovamente lo zombie con la velocità aggiornata.
A tal proposito è stata implementata una val \texttt{slowVelocities} dentro all'object ZombieDefaultValues.
E' infatti nell'object \texttt{ZombieDefaultValue} che sono definite tutte le caratteristiche principali per ogni
tipo di Zombie implementato: stato iniziale e di default (\texttt{Moving}), vita iniziale,
il tipo bullet sparato, la velocità di default, la velocità degli Zombie rallentati e un metodo per generare
la posizione iniziale degli zombie.

