\newpage
\section{Implementazione}


\subsection{Francesco Foschini}
Il mio compito all'interno del progetto, è stato quello di modellare la gerarchia di \texttt{Entity} del gioco.
In particolare mi sono occupato della implementazione del model degli Zombie, e i proiettili.
Ho partecipato inoltre allo sviluppo del Troop Actor del Menù Screen e del GameScreen.

\subsubsection{Bullet}
I proiettili rappresentano all'interno del gioco l'attacco delle piante e degli zombie che viene sparato dalle torri
per infliggere danni. Il trait \texttt{Bullet} è quindi rappresentato dall'estensione del
trait \texttt{Entity} a cui è stata aggiunta l'abilità di movimento (\texttt{Movement Ability}).
Tutti i bullet presentano un comportamento comune: possiedono una posizione, una velocità, una dimensione ed un danno.

Successivamente a una prima versione dei \texttt{Bullet} ho pensato di rifattorizzare i Bullet individuando due diversi sottotipi: \texttt{ZombieBullet} e \texttt{PlantBullet}.
Il primo è un trait che modella l'attacco base degli zombie mentre il secondo si riferisce all'attacco delle piante.
Attraverso il metodo checkCollision i PlantBullet possono collidere solamente contro le entità di tipo Zombie mentre,
al contrario, gli ZombieBullet potranno colpire solamente le entità sottotipo di pianta.
Mi sono successivamente concentrato nell'implementazione di due tipi di ZombieBullet: PawBullet (il bullet dei
BasicZombie e FastZombie) e lo SwordBullet(il bullet dei WarriorZombie).
La velocità e il danno di ogni tipo di bullet è definito nell'oggetto BulletDefaultValues.