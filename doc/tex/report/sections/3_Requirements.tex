\section{Requisiti}

\subsection{Funzionali}
Il gioco si compone di un insieme di entità e regol, per quanto riguarda gli elementi della partita:
\begin{itemize}
    \item le entità in gioco possono essere di tre tipi:
    \begin{enumerate}
        \item zombie
        \item piante
        \item bullet
    \end{enumerate}
    \item Il campo da gioco deve essere composto da più corsie.
    \item I nemici avanzeranno lungo le corsie da destra verso sinistra.
    \item Ogni troop deve avere un campo visivo.
    \item Quando uno zombie si avvicina sufficientemente ad una pianta, lo zombie la attaccherà.
    \item Quando un zombie entra nel campo visivo di una pianta, la pianta lo attaccherà.
    \item Se una troop viene colpita, perderà vita.
    \item Se una troop rimane senza vita, morirà e verra rimossa dal campo di gioco.
    \item Se uno zombie raggiunge la fine della corsia, la partita termina.
    \item I tipi di piante previsti sono:
    \begin{enumerate}
        \item Peashooter: spara bullet che colpiscono il primo zombie che incontrano.
        \item Wallnut: è molto resistente, ma non può attaccare.
        \item Cherrybomb: quando piazzata, genera un esplosione che danneggia tutte le troop nelle vicinanze.
    \end{enumerate}
    \item I tipi di zombie previsti sono:
    \begin{enumerate}
        \item BasicZombie: quando in contra una pianta, la attacca.
        \item WarriorZombie: ha molta vita e attacchi potenti, ma è più lento.
        \item FastZombie: è molto veloce.
    \end{enumerate}
    \item I tipi di bullet previsti sono:
    \begin{enumerate}
        \item Peabullet: il bullet sparato dal Peashooter.
        \item CherryBullet: il bullet sparato dalla Cherrybomb.
        \item PawBullet: il bullet sparato dal BasicZombie e dal FastZombie.
        \item SwordBullet: il bullet sparato dal WarriorZombie.
    \end{enumerate}
\end{itemize}

\subsubsection{Utente}
L'utente deve poter:
\begin{itemize}
    \item Avviare una partita.
    \item Piazzare le piante in punti liberi del campo da gioco.
    \item Vedere le statistiche a fine partita.
\end{itemize}

\subsubsection{Non Funzionali}
I requisiti non funzionali individuati per il progetto sono:
\begin{itemize}
    \item Realizzazione di software estendibile e rivisitabile: per raggiungere tale obiettivo sono stati individuati
    degli standard come un processo di sviluppo ad-hoc assieme ad una serie di strumenti ausiliari per il mantenimento
    della qualità del software;
    \item Realizzazione di un'interfaccia che renda l'esperienza di gioco intuitiva e piacevole.
    \item Realizzazione di un menù di inizio e fine partita.
\end{itemize}

\subsection{Opzionali}
\begin{enumerate}
    \item Aumentare i tipi di piante.
    \item Aumentare i tipi di zombie.
    \item Aumentare i tipi di bullet.
    \item Inserire delle animazioni.
    \item Inserire la possibilità di generare campi di gioco particolari.
\end{enumerate}

\subsection{Implementativi}
Il gioco deve essere sviluppato in Scala e deve dipendere da Librerie e altro linguaggi tipo LibGdx e TuProlog.
Il software prodotto deve essere testato con ScalaTest per garantire la manutenzione, la qualità e la corretta integrazione del codice dei vari componenti del team.